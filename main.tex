
% VLDB template version of 2020-08-03 enhances the ACM template, version 1.7.0:
% https://www.acm.org/publications/proceedings-template
% The ACM Latex guide provides further information about the ACM template

\documentclass[sigconf, nonacm]{acmart}

%% The following content must be adapted for the final version
% paper-specific
% \newcommand\vldbdoi{XX.XX/XXX.XX}
% \newcommand\vldbpages{XXX-XXX}
% issue-specific
% \newcommand\vldbvolume{14}
% \newcommand\vldbissue{1}
% \newcommand\vldbyear{2020}
% should be fine as it is
\newcommand\vldbauthors{\authors}
\newcommand\vldbtitle{\shorttitle} 
% leave empty if no availability url should be set
\newcommand\vldbavailabilityurl{https://github.com/kon-drees/RepEng-JSONSchemaDiscovery}
% whether page numbers should be shown or not, use 'plain' for review versions, 'empty' for camera ready
\newcommand\vldbpagestyle{plain} 

\begin{document}
\title{RepEng: Reproducibility of the Project JSON Schema Discovery}

%%
%% The "author" command and its associated commands are used to define the authors and their affiliations.
\author{Konrad Drees}
\affiliation{%
  \institution{University of Passau}
  \city{Passau}
  \state{Germany}
}
\email{drees03@ads.uni-passau.de}


\maketitle

%%% do not modify the following VLDB block %%
%%% VLDB block start %%%
\ifdefempty{\vldbavailabilityurl}{}{
\vspace{.3cm}
\begingroup\small\noindent\raggedright\textbf{Artifact Availability:}\\
The source code, data, and/or other artifacts have been made available at \url{\vldbavailabilityurl}.
\endgroup
}
%%% VLDB block end %%%

\section{Introduction}

This reproduction study looks at the research and experiments detailed in the paper "An Approach for Schema Extraction of JSON and Extended JSON Document Collections"\cite{Frozza2018}. The original study presented a methodology for schema discovery in NoSQL databases, with a focus on JSON and Extended JSON documents. NoSQL databases operate without a fixed schema due to their flexibility in data format handling \cite{Sadalage2013}.  Therefore  NoSQL databases have high availability and elasticity capabilities but also introduces challenges in data organisation, data analysis and retrieval \cite{Wischenbart2012}.  One of the NoSQL Databases are Document-oriented Databases, which store  and retrieve documents with simple and complex attributes mainly in JSON (JavaScript Object Notation) or Extended JSON formats\cite{Frozza2018}.
 This Project aims to methodically validate the  methodology and evaluates the reproducibility of the original paper by replicating the experiments.


\section{Replication of Processing Time Evaluation in Schema Extraction}

This reproduction study aims to replicate and confirm the findings of the original paper regarding the Processing Time Evaluation of the JSON Schema Discovery approach. Specifically, it  validates the reported efficiency of this approach in accurately extracting schemas from JSON documents within NoSQL databases. The focus will be on assessing whether the replication of the original experiments shows similar processing times, thereby confirming the method's reliability and efficiency as reported in the initial study. 


\subsection{Methodology }


\begin{table}[hb]
\centering
\caption{Results for Foursquare Datasets of the original Paper \cite{Frozza2018}. }
\label{table:results}
\resizebox{\columnwidth}{!}{\begin{tabular}{|l|r|r|r|c|c|c|}
\hline
\textbf{Collection} & \textbf{N\_JSON} & \textbf{RS} & \textbf{ROrd} & \textbf{TB} & \textbf{TT} & \textbf{TB/TT} \\
\hline
venues & 2 mil & 257 & 117 & 7,47 min & 7,52 min & 99,33\%  \\
checkins & 11 mil & 2 & 2 & 35,27 min & 35,52 min & 99,29\%  \\
tweets & 17 mil & 23 & 16 & 53,11 min & 53,44 min & 99,38\% \\
\hline
\end{tabular}
}
\end{table}

For evaluating the processing time of JSON Schema Discovery, this reproduction study will be conducted using a Docker container on a local machine. The experimental setup will replicate the original environment to the extent possible, given the hardware and software differences. The datasets used will be similar to those in the original study, including tracked data from tweets, check-ins, and venues from Foursquare. The Docker environment will ensure a controlled, consistent, and replicable setting for comparing processing times against those reported in the original study on an Amazon EC2 instance





Table \ref{table:results} shows the  results of the experiments of the original study \cite{Frozza2018}.
The criteria for confirming the experiment involve matching the original study's reported ratio \textbf{TB/TT} between Time to Obtain the raw schemas (TB) and Total Time (TT). A similar  TB/TT ratio would indicate successful replication of processing efficiency. 


\subsection{Analysis of the Reproducibility Process}

In order to evaluate reproducibility of the original study, a replication package will be created.  The analysis contains the methodological steps taken and the challenges encountered, such as obtaining equivalent datasets, running the NoSQL Database, ensuring computational reproducibility, and using the original codebase within a local Docker environment.

The analysis will also evaluate the documentation and artefact availability from the original study. 


%\clearpage

\bibliographystyle{ACM-Reference-Format}
\bibliography{main}

\end{document}
\endinput
